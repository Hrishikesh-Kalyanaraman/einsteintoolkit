
\section{Schedule} 


\parskip = 0pt


\noindent This section lists all the variables which are assigned storage by thorn EinsteinEvolve/GRHydro\_InitData.  Storage can either last for the duration of the run ({\bf Always} means that if this thorn is activated storage will be assigned, {\bf Conditional} means that if this thorn is activated storage will be assigned for the duration of the run if some condition is met), or can be turned on for the duration of a schedule function.


\subsection*{Storage}

\hspace{5mm}

 \begin{tabular*}{160mm}{ll} 
~& {\bf Conditional:} \\ 
~ &  GRHydro\_init\_data\_reflevel\\ 
~ &  GRHydro\_init\_data\_reflevel\\ 
~ &  GRHydro\_init\_data\_reflevel\\ 
~ &  GRHydro\_init\_data\_reflevel\\ 
~ &  simple\_wave\_grid\_functions\\ 
~ &  simple\_wave\_scalars\\ 
~ &  simple\_wave\_output\\ 
~ & ~\\ 
\end{tabular*} 


\subsection*{Scheduled Functions}
\vspace{5mm}

\noindent {\bf CCTK\_PARAMCHECK} 

\hspace{5mm} grhydro\_initdata\_checkparameters 

\hspace{5mm}{\it check parameters } 


\hspace{5mm}

 \begin{tabular*}{160mm}{cll} 
~ & Language:  & c \\ 
~ & Type:  & function \\ 
\end{tabular*} 


\vspace{5mm}

\noindent {\bf HydroBase\_Initial}   (conditional) 

\hspace{5mm} grhydro\_monopolem 

\hspace{5mm}{\it monopole initial data } 


\hspace{5mm}

 \begin{tabular*}{160mm}{cll} 
~ & Language:  & fortran \\ 
~ & Type:  & function \\ 
\end{tabular*} 


\vspace{5mm}

\noindent {\bf HydroBase\_Initial}   (conditional) 

\hspace{5mm} grhydro\_shocktube 

\hspace{5mm}{\it shocktube initial data } 


\hspace{5mm}

 \begin{tabular*}{160mm}{cll} 
~ & Language:  & fortran \\ 
~ & Type:  & function \\ 
\end{tabular*} 


\vspace{5mm}

\noindent {\bf HydroBase\_Initial}   (conditional) 

\hspace{5mm} grhydro\_cylindricalexplosionm 

\hspace{5mm}{\it cylindrical explosion initial data - mhd-only } 


\hspace{5mm}

 \begin{tabular*}{160mm}{cll} 
~ & Language:  & fortran \\ 
~ & Type:  & function \\ 
\end{tabular*} 


\vspace{5mm}

\noindent {\bf HydroBase\_Initial}   (conditional) 

\hspace{5mm} grhydro\_init\_data\_refinementlevel 

\hspace{5mm}{\it calculate current refinement level } 


\hspace{5mm}

 \begin{tabular*}{160mm}{cll} 
~ & Before:  & grhydro\_con2primtest \\ 
~ & Language:  & fortran \\ 
~ & Type:  & function \\ 
\end{tabular*} 


\vspace{5mm}

\noindent {\bf HydroBase\_Initial}   (conditional) 

\hspace{5mm} grhydro\_con2primtest 

\hspace{5mm}{\it testing the conservative to primitive solver } 


\hspace{5mm}

 \begin{tabular*}{160mm}{cll} 
~ & Language:  & fortran \\ 
~ & Type:  & function \\ 
\end{tabular*} 


\vspace{5mm}

\noindent {\bf HydroBase\_Initial}   (conditional) 

\hspace{5mm} grhydro\_init\_data\_refinementlevel 

\hspace{5mm}{\it calculate current refinement level } 


\hspace{5mm}

 \begin{tabular*}{160mm}{cll} 
~ & Before:  & c2p2c\_call \\ 
~ & Language:  & fortran \\ 
~ & Type:  & function \\ 
\end{tabular*} 


\vspace{5mm}

\noindent {\bf HydroBase\_Initial}   (conditional) 

\hspace{5mm} c2p2cm 

\hspace{5mm}{\it testing conservative to primitive to conservative - mhd version } 


\hspace{5mm}

 \begin{tabular*}{160mm}{cll} 
~ & Language:  & fortran \\ 
~ & Type:  & function \\ 
\end{tabular*} 


\vspace{5mm}

\noindent {\bf HydroBase\_Initial}   (conditional) 

\hspace{5mm} c2p2c 

\hspace{5mm}{\it testing conservative to primitive to conservative } 


\hspace{5mm}

 \begin{tabular*}{160mm}{cll} 
~ & Language:  & fortran \\ 
~ & Type:  & function \\ 
\end{tabular*} 


\vspace{5mm}

\noindent {\bf HydroBase\_Initial}   (conditional) 

\hspace{5mm} grhydro\_init\_data\_refinementlevel 

\hspace{5mm}{\it calculate current refinement level } 


\hspace{5mm}

 \begin{tabular*}{160mm}{cll} 
~ & Before:  & p2c2p\_call \\ 
~ & Language:  & fortran \\ 
~ & Type:  & function \\ 
\end{tabular*} 


\vspace{5mm}

\noindent {\bf HydroBase\_Initial}   (conditional) 

\hspace{5mm} p2c2pm 

\hspace{5mm}{\it testing primitive to conservative to primitive - mhd version } 


\hspace{5mm}

 \begin{tabular*}{160mm}{cll} 
~ & Language:  & fortran \\ 
~ & Type:  & function \\ 
\end{tabular*} 


\vspace{5mm}

\noindent {\bf HydroBase\_Initial}   (conditional) 

\hspace{5mm} p2c2p 

\hspace{5mm}{\it testing primitive to conservative to primitive } 


\hspace{5mm}

 \begin{tabular*}{160mm}{cll} 
~ & Language:  & fortran \\ 
~ & Type:  & function \\ 
\end{tabular*} 


\vspace{5mm}

\noindent {\bf HydroBase\_Initial}   (conditional) 

\hspace{5mm} grhydro\_rotorm 

\hspace{5mm}{\it mhd rotor initial data } 


\hspace{5mm}

 \begin{tabular*}{160mm}{cll} 
~ & Language:  & fortran \\ 
~ & Type:  & function \\ 
\end{tabular*} 


\vspace{5mm}

\noindent {\bf HydroBase\_Initial}   (conditional) 

\hspace{5mm} grhydro\_init\_data\_refinementlevel 

\hspace{5mm}{\it calculate current refinement level } 


\hspace{5mm}

 \begin{tabular*}{160mm}{cll} 
~ & Before:  & p2c2p\_call \\ 
~ & Language:  & fortran \\ 
~ & Type:  & function \\ 
\end{tabular*} 


\vspace{5mm}

\noindent {\bf HydroBase\_Initial}   (conditional) 

\hspace{5mm} p2c2pm\_polytype 

\hspace{5mm}{\it testing primitive to conservative to primitive - mhd polytype version } 


\hspace{5mm}

 \begin{tabular*}{160mm}{cll} 
~ & Language:  & fortran \\ 
~ & Type:  & function \\ 
\end{tabular*} 


\vspace{5mm}

\noindent {\bf HydroBase\_Initial}   (conditional) 

\hspace{5mm} grhydro\_reconstruction\_test 

\hspace{5mm}{\it testing the reconstruction } 


\hspace{5mm}

 \begin{tabular*}{160mm}{cll} 
~ & Language:  & fortran \\ 
~ & Options:  & global \\ 
~& ~ &loop-local\\ 
~ & Storage:  & grhydro\_prim\_bext \\ 
~& ~ &grhydro\_scalars\\ 
~ & Type:  & function \\ 
\end{tabular*} 


\vspace{5mm}

\noindent {\bf HydroBase\_Initial}   (conditional) 

\hspace{5mm} grhydro\_only\_atmo 

\hspace{5mm}{\it only atmosphere as initial data } 


\hspace{5mm}

 \begin{tabular*}{160mm}{cll} 
~ & Language:  & fortran \\ 
~ & Type:  & function \\ 
\end{tabular*} 


\vspace{5mm}

\noindent {\bf HydroBase\_Initial}   (conditional) 

\hspace{5mm} grhydro\_readconformaldata 

\hspace{5mm}{\it set the missing quantities, after reading in from file initial data from conformally-flat codes (garching) } 


\hspace{5mm}

 \begin{tabular*}{160mm}{cll} 
~ & Language:  & fortran \\ 
~ & Type:  & function \\ 
\end{tabular*} 


\vspace{5mm}

\noindent {\bf HydroBase\_Initial}   (conditional) 

\hspace{5mm} grhydro\_simplewave 

\hspace{5mm}{\it set initial data from anile miller motta, phys.fluids. 26, 1450 (1983) } 


\hspace{5mm}

 \begin{tabular*}{160mm}{cll} 
~ & Language:  & fortran \\ 
~ & Type:  & function \\ 
\end{tabular*} 


\vspace{5mm}

\noindent {\bf CCTK\_ANALYSIS}   (conditional) 

\hspace{5mm} grhydro\_simplewave\_analysis 

\hspace{5mm}{\it compute some output variables for the simple wave } 


\hspace{5mm}

 \begin{tabular*}{160mm}{cll} 
~ & After:  & grhydro\_entropy \\ 
~ & Language:  & fortran \\ 
~ & Type:  & function \\ 
\end{tabular*} 


\vspace{5mm}

\noindent {\bf HydroBase\_Initial}   (conditional) 

\hspace{5mm} grhydro\_bondi\_iso 

\hspace{5mm}{\it setup grhydro vars for the hydrodynamic bondi solution } 


\hspace{5mm}

 \begin{tabular*}{160mm}{cll} 
~ & After:  & hydrobase\_excisionmasksetup \\ 
~ & Language:  & fortran \\ 
~ & Type:  & function \\ 
\end{tabular*} 


\vspace{5mm}

\noindent {\bf HydroBase\_Initial}   (conditional) 

\hspace{5mm} grhydro\_bondim\_iso 

\hspace{5mm}{\it setup grhydro vars for the magnetized bondi solution } 


\hspace{5mm}

 \begin{tabular*}{160mm}{cll} 
~ & After:  & hydrobase\_excisionmasksetup \\ 
~ & Language:  & fortran \\ 
~ & Type:  & function \\ 
\end{tabular*} 


\vspace{5mm}

\noindent {\bf HydroBase\_Initial}   (conditional) 

\hspace{5mm} grhydro\_bondi 

\hspace{5mm}{\it setup grhydro vars for the hydrodynamic bondi solution } 


\hspace{5mm}

 \begin{tabular*}{160mm}{cll} 
~ & After:  & hydrobase\_excisionmasksetup \\ 
~ & Language:  & c \\ 
~ & Type:  & function \\ 
\end{tabular*} 


\vspace{5mm}

\noindent {\bf HydroBase\_Initial}   (conditional) 

\hspace{5mm} grhydro\_advectedloopm 

\hspace{5mm}{\it mhd advected loop initial data } 


\hspace{5mm}

 \begin{tabular*}{160mm}{cll} 
~ & Language:  & fortran \\ 
~ & Type:  & function \\ 
\end{tabular*} 


\vspace{5mm}

\noindent {\bf HydroBase\_Initial}   (conditional) 

\hspace{5mm} grhydro\_bondim 

\hspace{5mm}{\it setup grhydro vars for the magnetized bondi solution } 


\hspace{5mm}

 \begin{tabular*}{160mm}{cll} 
~ & After:  & hydrobase\_excisionmasksetup \\ 
~ & Language:  & c \\ 
~ & Type:  & function \\ 
\end{tabular*} 


\vspace{5mm}

\noindent {\bf HydroBase\_Con2Prim}   (conditional) 

\hspace{5mm} grhydro\_bondim\_range 

\hspace{5mm}{\it force analytic solution outside anulus } 


\hspace{5mm}

 \begin{tabular*}{160mm}{cll} 
~ & Before:  & con2prim \\ 
~ & Language:  & c \\ 
~ & Type:  & function \\ 
\end{tabular*} 


\vspace{5mm}

\noindent {\bf HydroBase\_Boundaries}   (conditional) 

\hspace{5mm} grhydro\_bondim\_boundary 

\hspace{5mm}{\it force analytic solution in boundaries } 


\hspace{5mm}

 \begin{tabular*}{160mm}{cll} 
~ & Before:  & hydrobase\_select\_boundaries \\ 
~ & Language:  & c \\ 
~ & Type:  & function \\ 
\end{tabular*} 


\vspace{5mm}

\noindent {\bf CCTK\_INITIAL}   (conditional) 

\hspace{5mm} grhydro\_poloidalmagfieldm 

\hspace{5mm}{\it set up a poloidal magnetic field. it expects the other fluid variables already to be set, as for example in the tov solution } 


\hspace{5mm}

 \begin{tabular*}{160mm}{cll} 
~ & After:  & hydrobase\_initial \\ 
~ & Before:  & grhydrotransformprimtolocalbasis \\ 
~ & Language:  & fortran \\ 
~ & Type:  & function \\ 
\end{tabular*} 


\vspace{5mm}

\noindent {\bf CCTK\_INITIAL}   (conditional) 

\hspace{5mm} hydrobase\_boundaries 

\hspace{5mm}{\it call boundary conditions after magnetic field initial data setup } 


\hspace{5mm}

 \begin{tabular*}{160mm}{cll} 
~ & After:  & grhydro\_poloidalmagfieldm \\ 
~ & Before:  & grhydrotransformprimtolocalbasis \\ 
~ & Type:  & group \\ 
\end{tabular*} 


\vspace{5mm}

\noindent {\bf HydroBase\_Initial}   (conditional) 

\hspace{5mm} grhydro\_alfvenwavem 

\hspace{5mm}{\it circularly polarized alfven wave initial data } 


\hspace{5mm}

 \begin{tabular*}{160mm}{cll} 
~ & Language:  & fortran \\ 
~ & Type:  & function \\ 
\end{tabular*} 


\vspace{5mm}

\noindent {\bf HydroBase\_Initial}   (conditional) 

\hspace{5mm} grhydro\_shocktube\_hot 

\hspace{5mm}{\it hot shocktube initial data } 


\hspace{5mm}

 \begin{tabular*}{160mm}{cll} 
~ & After:  & hydrobase\_y\_e\_one \\ 
~& ~ &hydrobase\_zero\\ 
~ & Language:  & fortran \\ 
~ & Type:  & function \\ 
\end{tabular*} 


\vspace{5mm}

\noindent {\bf HydroBase\_Initial}   (conditional) 

\hspace{5mm} grhydro\_shocktubem 

\hspace{5mm}{\it shocktube initial data - mhd version } 


\hspace{5mm}

 \begin{tabular*}{160mm}{cll} 
~ & Language:  & fortran \\ 
~ & Type:  & function \\ 
\end{tabular*} 


\vspace{5mm}

\noindent {\bf ApplyBCs}   (conditional) 

\hspace{5mm} grhydro\_diagshock\_boundarym 

\hspace{5mm}{\it diagonal shock boundary conditions } 


\hspace{5mm}

 \begin{tabular*}{160mm}{cll} 
~ & After:  & boundaryconditions \\ 
~& ~ &boundary::boundary\_clearselection\\ 
~ & Language:  & fortran \\ 
~ & Sync:  & grhydro::dens \\ 
~& ~ &grhydro::tau\\ 
~& ~ &grhydro::scon\\ 
~& ~ &grhydro::bcons\\ 
~& ~ &grhydro::psidc\\ 
~ & Type:  & function \\ 
\end{tabular*} 


\vspace{5mm}

\noindent {\bf ApplyBCs}   (conditional) 

\hspace{5mm} grhydro\_diagshock\_boundarym 

\hspace{5mm}{\it diagonal shock boundary conditions } 


\hspace{5mm}

 \begin{tabular*}{160mm}{cll} 
~ & After:  & boundaryconditions \\ 
~& ~ &boundary::boundary\_clearselection\\ 
~ & Language:  & fortran \\ 
~ & Sync:  & grhydro::dens \\ 
~& ~ &grhydro::tau\\ 
~& ~ &grhydro::scon\\ 
~& ~ &grhydro::bcons\\ 
~ & Type:  & function \\ 
\end{tabular*} 


\vspace{5mm}

\noindent {\bf ApplyBCs}   (conditional) 

\hspace{5mm} grhydro\_diagshock2d\_boundarym 

\hspace{5mm}{\it 2-d diagonal shock boundary conditions } 


\hspace{5mm}

 \begin{tabular*}{160mm}{cll} 
~ & After:  & boundaryconditions \\ 
~& ~ &boundary::boundary\_clearselection\\ 
~ & Language:  & fortran \\ 
~ & Type:  & function \\ 
\end{tabular*} 


\subsection*{Aliased Functions}

\hspace{5mm}

 \begin{tabular*}{160mm}{ll} 

{\bf Alias Name:} ~~~~~~~ & {\bf Function Name:} \\ 
c2p2c & c2p2c\_call \\ 
c2p2cM & c2p2c\_call \\ 
p2c2p & p2c2p\_call \\ 
p2c2pM & p2c2p\_call \\ 
p2c2pM\_polytype & p2c2p\_call \\ 
\end{tabular*} 



\vspace{5mm}\parskip = 10pt 
